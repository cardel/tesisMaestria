%% Generated by Sphinx.
\def\sphinxdocclass{report}
\documentclass[letterpaper,10pt,english]{sphinxmanual}
\ifdefined\pdfpxdimen
   \let\sphinxpxdimen\pdfpxdimen\else\newdimen\sphinxpxdimen
\fi \sphinxpxdimen=.75bp\relax

\PassOptionsToPackage{warn}{textcomp}
\usepackage[utf8]{inputenc}
\ifdefined\DeclareUnicodeCharacter
% support both utf8 and utf8x syntaxes
\edef\sphinxdqmaybe{\ifdefined\DeclareUnicodeCharacterAsOptional\string"\fi}
  \DeclareUnicodeCharacter{\sphinxdqmaybe00A0}{\nobreakspace}
  \DeclareUnicodeCharacter{\sphinxdqmaybe2500}{\sphinxunichar{2500}}
  \DeclareUnicodeCharacter{\sphinxdqmaybe2502}{\sphinxunichar{2502}}
  \DeclareUnicodeCharacter{\sphinxdqmaybe2514}{\sphinxunichar{2514}}
  \DeclareUnicodeCharacter{\sphinxdqmaybe251C}{\sphinxunichar{251C}}
  \DeclareUnicodeCharacter{\sphinxdqmaybe2572}{\textbackslash}
\fi
\usepackage{cmap}
\usepackage[T1]{fontenc}
\usepackage{amsmath,amssymb,amstext}
\usepackage{babel}
\usepackage{times}
\usepackage[Bjarne]{fncychap}
\usepackage{sphinx}

\fvset{fontsize=\small}
\usepackage{geometry}

% Include hyperref last.
\usepackage{hyperref}
% Fix anchor placement for figures with captions.
\usepackage{hypcap}% it must be loaded after hyperref.
% Set up styles of URL: it should be placed after hyperref.
\urlstyle{same}

\addto\captionsenglish{\renewcommand{\figurename}{Fig.\@ }}
\makeatletter
\def\fnum@figure{\figurename\thefigure{}}
\makeatother
\addto\captionsenglish{\renewcommand{\tablename}{Table }}
\makeatletter
\def\fnum@table{\tablename\thetable{}}
\makeatother
\addto\captionsenglish{\renewcommand{\literalblockname}{Listing}}

\addto\captionsenglish{\renewcommand{\literalblockcontinuedname}{continued from previous page}}
\addto\captionsenglish{\renewcommand{\literalblockcontinuesname}{continues on next page}}
\addto\captionsenglish{\renewcommand{\sphinxnonalphabeticalgroupname}{Non-alphabetical}}
\addto\captionsenglish{\renewcommand{\sphinxsymbolsname}{Symbols}}
\addto\captionsenglish{\renewcommand{\sphinxnumbersname}{Numbers}}

\addto\extrasenglish{\def\pageautorefname{page}}

\setcounter{tocdepth}{3}
\setcounter{secnumdepth}{3}


\title{LAMFRIA Documentation}
\date{Oct 07, 2020}
\release{1}
\author{Carlos Andres Delgado S}
\newcommand{\sphinxlogo}{\vbox{}}
\renewcommand{\releasename}{Release}
\makeindex
\begin{document}

\pagestyle{empty}
\sphinxmaketitle
\pagestyle{plain}
\sphinxtableofcontents
\pagestyle{normal}
\phantomsection\label{\detokenize{index::doc}}



\chapter{Description}
\label{\detokenize{index:description}}
LAMFRIA is a tool for multifractal and robustness analisys. It also includes two artificial intelligence strategies for calculate fractal dimensions.

Contents:


\section{Box Counting algorithm}
\label{\detokenize{BCAlgorithm:box-counting-algorithm}}\label{\detokenize{BCAlgorithm::doc}}
This package provides multifractal analysis with Box Counting Algorithm proposed in journal article: Fractal and multifractal properties of a family of fractal networks DOI: 10.1088/1742-5468/2014/02/P02020


\subsection{BoxCounting algorithm:}
\label{\detokenize{BCAlgorithm:module-BCAlgorithm.BCAlgorithm}}\label{\detokenize{BCAlgorithm:boxcounting-algorithm}}\index{BCAlgorithm.BCAlgorithm (module)@\spxentry{BCAlgorithm.BCAlgorithm}\spxextra{module}}\index{BCAlgorithm() (in module BCAlgorithm.BCAlgorithm)@\spxentry{BCAlgorithm()}\spxextra{in module BCAlgorithm.BCAlgorithm}}

\begin{fulllineitems}
\phantomsection\label{\detokenize{BCAlgorithm:BCAlgorithm.BCAlgorithm.BCAlgorithm}}\pysiglinewithargsret{\sphinxcode{\sphinxupquote{BCAlgorithm.BCAlgorithm.}}\sphinxbfcode{\sphinxupquote{BCAlgorithm}}}{\emph{g}, \emph{minq}, \emph{maxq}, \emph{percentNodesT}, \emph{centerNodes=array({[}{]}}, \emph{dtype=float64)}}{}
Calculate fractal dimension with BoxCounting method

Inputs are parameters to configure algorithm behaviour.
\begin{quote}\begin{description}
\item[{Parameters}] \leavevmode\begin{itemize}
\item {} 
\sphinxstyleliteralstrong{\sphinxupquote{g}} (\sphinxstyleliteralemphasis{\sphinxupquote{Snap PUN Graph.}}) \textendash{} Network.

\item {} 
\sphinxstyleliteralstrong{\sphinxupquote{minq}} \textendash{} Minimum value of q

\item {} 
\sphinxstyleliteralstrong{\sphinxupquote{minq}} \textendash{} Maximum value of q

\item {} 
\sphinxstyleliteralstrong{\sphinxupquote{percentNodesT}} (\sphinxstyleliteralemphasis{\sphinxupquote{Integer}}) \textendash{} Number of combinations of center nodes. This value is a percent of the total nodes

\item {} 
\sphinxstyleliteralstrong{\sphinxupquote{CenterNodes}} (\sphinxstyleliteralemphasis{\sphinxupquote{Numpy 1D Array}}) \textendash{} Calculated center. If this is null, then the centers are calculated

\end{itemize}

\item[{Returns}] \leavevmode
\begin{description}
\item[{logR: Numpy array}] \leavevmode
logarithm of r/d

\item[{Indexzero: Integer}] \leavevmode
position of q=0 in Tq and Dq

\item[{Tq: Numpy array}] \leavevmode
mass exponents

\item[{Dq: Numpy array}] \leavevmode
fractal dimensions

\item[{lnMrq: Numpy 2D array}] \leavevmode
logarithm of number of nodes in boxes by radio

\end{description}


\end{description}\end{quote}

\end{fulllineitems}



\subsection{Example}
\label{\detokenize{BCAlgorithm:example}}
\fvset{hllines={, 2, 7,}}%
\begin{sphinxVerbatim}[commandchars=\\\{\}]
\PYG{k+kn}{import} \PYG{n+nn}{sys}
\PYG{k+kn}{import} \PYG{n+nn}{lib.snap} \PYG{k+kn}{as} \PYG{n+nn}{snap}
\PYG{k+kn}{import} \PYG{n+nn}{BCAlgorithm.BCAlgorithm} \PYG{k+kn}{as} \PYG{n+nn}{BCAlgorithm}
\PYG{k+kn}{import} \PYG{n+nn}{numpy}

\PYG{n}{minq} \PYG{o}{=} \PYG{o}{\PYGZhy{}}\PYG{l+m+mi}{10}
\PYG{n}{maxq} \PYG{o}{=} \PYG{l+m+mi}{10}
\PYG{n}{percentNodesT} \PYG{o}{=} \PYG{l+m+mi}{2} \PYG{c+c1}{\PYGZsh{}200\PYGZpc{} of nodes}
\PYG{n}{Rnd} \PYG{o}{=} \PYG{n}{snap}\PYG{o}{.}\PYG{n}{TRnd}\PYG{p}{(}\PYG{l+m+mi}{1}\PYG{p}{,}\PYG{l+m+mi}{0}\PYG{p}{)}
\PYG{n}{graph} \PYG{o}{=} \PYG{n}{snap}\PYG{o}{.}\PYG{n}{GenPrefAttach}\PYG{p}{(}\PYG{l+m+mi}{10000}\PYG{p}{,} \PYG{l+m+mi}{10}\PYG{p}{,}\PYG{n}{Rnd}\PYG{p}{)}  \PYG{c+c1}{\PYGZsh{}ScaleFree with 10 edges per node}

\PYG{n}{logR}\PYG{p}{,} \PYG{n}{Indexzero}\PYG{p}{,}\PYG{n}{Tq}\PYG{p}{,} \PYG{n}{Dq}\PYG{p}{,} \PYG{n}{lnMrq} \PYG{o}{=} \PYG{n}{BCAlgorithm}\PYG{o}{.}\PYG{n}{BCAlgorithm}\PYG{p}{(}\PYG{n}{graph}\PYG{p}{,}\PYG{n}{minq}\PYG{p}{,}\PYG{n}{maxq}\PYG{p}{,}\PYG{n}{percentNodesT}\PYG{p}{)}
\end{sphinxVerbatim}
\sphinxresetverbatimhllines


\section{Fixed Size Box Counting Algorithm}
\label{\detokenize{FSBCAlgorithm:fixed-size-box-counting-algorithm}}\label{\detokenize{FSBCAlgorithm::doc}}
This package provides multifractal analysis with Box Counting Fixed Size Algorithm proposed in journal article: Multifractal analysis of complex networks DOI: 10.1088/1674-1056/21/8/080504


\subsection{Box Counting Fixed Size module}
\label{\detokenize{FSBCAlgorithm:module-FSBCAlgorithm.FSBCAlgorithm}}\label{\detokenize{FSBCAlgorithm:box-counting-fixed-size-module}}\index{FSBCAlgorithm.FSBCAlgorithm (module)@\spxentry{FSBCAlgorithm.FSBCAlgorithm}\spxextra{module}}\index{FSBCAlgorithm() (in module FSBCAlgorithm.FSBCAlgorithm)@\spxentry{FSBCAlgorithm()}\spxextra{in module FSBCAlgorithm.FSBCAlgorithm}}

\begin{fulllineitems}
\phantomsection\label{\detokenize{FSBCAlgorithm:FSBCAlgorithm.FSBCAlgorithm.FSBCAlgorithm}}\pysiglinewithargsret{\sphinxcode{\sphinxupquote{FSBCAlgorithm.FSBCAlgorithm.}}\sphinxbfcode{\sphinxupquote{FSBCAlgorithm}}}{\emph{g}, \emph{minq}, \emph{maxq}, \emph{percentNodesT}, \emph{centerNodes=array({[}{]}}, \emph{dtype=float64)}}{}
Calculate fractal dimension with BoxCounting fixed size method

Inputs are parameters to configure algorithm behaviour.
\begin{quote}\begin{description}
\item[{Parameters}] \leavevmode\begin{itemize}
\item {} 
\sphinxstyleliteralstrong{\sphinxupquote{g}} (\sphinxstyleliteralemphasis{\sphinxupquote{Snap PUN Graph.}}) \textendash{} Network.

\item {} 
\sphinxstyleliteralstrong{\sphinxupquote{minq}} \textendash{} Minimum value of q

\item {} 
\sphinxstyleliteralstrong{\sphinxupquote{minq}} \textendash{} Maximum value of q

\item {} 
\sphinxstyleliteralstrong{\sphinxupquote{percentNodesT}} \textendash{} Number of combinations of center nodes. This value is a percent of the total nodes

\item {} 
\sphinxstyleliteralstrong{\sphinxupquote{percentNodesT}} \textendash{} Center of boxes. All centers have to different and the lenght array must be equal to number of nodes

\end{itemize}

\item[{Returns}] \leavevmode
\begin{description}
\item[{logR: Numpy array}] \leavevmode
logarithm of r/d

\item[{Indexzero: Integer}] \leavevmode
position of q=0 in Tq and Dq

\item[{Tq: Numpy array}] \leavevmode
mass exponents

\item[{Dq: Numpy array}] \leavevmode
fractal dimensions

\item[{lnMrq: Numpy 2D array}] \leavevmode
logarithm of number of nodes in boxes by radio

\end{description}


\end{description}\end{quote}

\end{fulllineitems}



\subsection{Example}
\label{\detokenize{FSBCAlgorithm:example}}
\fvset{hllines={, 2, 7,}}%
\begin{sphinxVerbatim}[commandchars=\\\{\}]
\PYG{k+kn}{import} \PYG{n+nn}{sys}
\PYG{k+kn}{import} \PYG{n+nn}{lib.snap} \PYG{k+kn}{as} \PYG{n+nn}{snap}
\PYG{k+kn}{import} \PYG{n+nn}{FSBCAlgorithm.FSBCAlgorithm} \PYG{k+kn}{as} \PYG{n+nn}{FSBCAlgorithm}
\PYG{k+kn}{import} \PYG{n+nn}{numpy}

\PYG{n}{minq} \PYG{o}{=} \PYG{o}{\PYGZhy{}}\PYG{l+m+mi}{10}
\PYG{n}{maxq} \PYG{o}{=} \PYG{l+m+mi}{10}
\PYG{n}{percentNodesT} \PYG{o}{=} \PYG{l+m+mi}{2} \PYG{c+c1}{\PYGZsh{}200\PYGZpc{} of nodes}
\PYG{n}{Rnd} \PYG{o}{=} \PYG{n}{snap}\PYG{o}{.}\PYG{n}{TRnd}\PYG{p}{(}\PYG{l+m+mi}{1}\PYG{p}{,}\PYG{l+m+mi}{0}\PYG{p}{)}
\PYG{n}{graph} \PYG{o}{=} \PYG{n}{snap}\PYG{o}{.}\PYG{n}{GenPrefAttach}\PYG{p}{(}\PYG{l+m+mi}{10000}\PYG{p}{,} \PYG{l+m+mi}{10}\PYG{p}{,}\PYG{n}{Rnd}\PYG{p}{)}  \PYG{c+c1}{\PYGZsh{}ScaleFree with 10 edges per node}

\PYG{n}{logR}\PYG{p}{,} \PYG{n}{Indexzero}\PYG{p}{,}\PYG{n}{Tq}\PYG{p}{,} \PYG{n}{Dq}\PYG{p}{,} \PYG{n}{lnMrq} \PYG{o}{=} \PYG{n}{FSBCAlgorithm}\PYG{o}{.}\PYG{n}{FSBCAlgorithm}\PYG{p}{(}\PYG{n}{graph}\PYG{p}{,}\PYG{n}{minq}\PYG{p}{,}\PYG{n}{maxq}\PYG{p}{,}\PYG{n}{percentNodesT}\PYG{p}{)}
\end{sphinxVerbatim}
\sphinxresetverbatimhllines


\section{CBBAlgorithm package}
\label{\detokenize{CBBAlgorithm:cbbalgorithm-package}}\label{\detokenize{CBBAlgorithm::doc}}

\subsection{Submodules}
\label{\detokenize{CBBAlgorithm:submodules}}

\subsection{CBBAlgorithm.CBBAlgorithm module}
\label{\detokenize{CBBAlgorithm:module-CBBAlgorithm.CBBAlgorithm}}\label{\detokenize{CBBAlgorithm:cbbalgorithm-cbbalgorithm-module}}\index{CBBAlgorithm.CBBAlgorithm (module)@\spxentry{CBBAlgorithm.CBBAlgorithm}\spxextra{module}}\index{CBBFractality() (in module CBBAlgorithm.CBBAlgorithm)@\spxentry{CBBFractality()}\spxextra{in module CBBAlgorithm.CBBAlgorithm}}

\begin{fulllineitems}
\phantomsection\label{\detokenize{CBBAlgorithm:CBBAlgorithm.CBBAlgorithm.CBBFractality}}\pysiglinewithargsret{\sphinxcode{\sphinxupquote{CBBAlgorithm.CBBAlgorithm.}}\sphinxbfcode{\sphinxupquote{CBBFractality}}}{\emph{graph}}{}
\end{fulllineitems}

\index{calculateLb() (in module CBBAlgorithm.CBBAlgorithm)@\spxentry{calculateLb()}\spxextra{in module CBBAlgorithm.CBBAlgorithm}}

\begin{fulllineitems}
\phantomsection\label{\detokenize{CBBAlgorithm:CBBAlgorithm.CBBAlgorithm.calculateLb}}\pysiglinewithargsret{\sphinxcode{\sphinxupquote{CBBAlgorithm.CBBAlgorithm.}}\sphinxbfcode{\sphinxupquote{calculateLb}}}{\emph{boxes}}{}
\end{fulllineitems}



\subsection{Module contents}
\label{\detokenize{CBBAlgorithm:module-CBBAlgorithm}}\label{\detokenize{CBBAlgorithm:module-contents}}\index{CBBAlgorithm (module)@\spxentry{CBBAlgorithm}\spxextra{module}}

\section{Genetic strategy}
\label{\detokenize{Genetic:genetic-strategy}}\label{\detokenize{Genetic::doc}}
Genetic algorithm for multifractal analysis


\subsection{Genetic}
\label{\detokenize{Genetic:module-Genetic.Genetic}}\label{\detokenize{Genetic:genetic}}\index{Genetic.Genetic (module)@\spxentry{Genetic.Genetic}\spxextra{module}}\index{Genetic() (in module Genetic.Genetic)@\spxentry{Genetic()}\spxextra{in module Genetic.Genetic}}

\begin{fulllineitems}
\phantomsection\label{\detokenize{Genetic:Genetic.Genetic.Genetic}}\pysiglinewithargsret{\sphinxcode{\sphinxupquote{Genetic.Genetic.}}\sphinxbfcode{\sphinxupquote{Genetic}}}{\emph{g}, \emph{minq}, \emph{maxq}, \emph{sizePopulation}, \emph{iterations}, \emph{percentCrossOver}, \emph{percentMutation}, \emph{degreeOfBoring}, \emph{typeAlgorithm}}{}
The genetic algorithm
\begin{quote}\begin{description}
\item[{Parameters}] \leavevmode\begin{itemize}
\item {} 
\sphinxstyleliteralstrong{\sphinxupquote{graph}} (\sphinxstyleliteralemphasis{\sphinxupquote{Snap PUN Graph.}}) \textendash{} Network.

\item {} 
\sphinxstyleliteralstrong{\sphinxupquote{iterations}} (\sphinxstyleliteralemphasis{\sphinxupquote{Integer}}) \textendash{} Number of max iterations

\item {} 
\sphinxstyleliteralstrong{\sphinxupquote{sizePopulation}} (\sphinxstyleliteralemphasis{\sphinxupquote{Integer}}) \textendash{} Size of population

\item {} 
\sphinxstyleliteralstrong{\sphinxupquote{percentCrossOver}} (\sphinxstyleliteralemphasis{\sphinxupquote{Double}}) \textendash{} Percent (0 to 1) of individual select to cross

\item {} 
\sphinxstyleliteralstrong{\sphinxupquote{percentMutation}} (\sphinxstyleliteralemphasis{\sphinxupquote{Double}}) \textendash{} (0 to 1) of probability to apply mutation to an individual

\item {} 
\sphinxstyleliteralstrong{\sphinxupquote{degreeOfBoring}} (\sphinxstyleliteralemphasis{\sphinxupquote{Integer}}) \textendash{} Number of iterations of boring

\item {} 
\sphinxstyleliteralstrong{\sphinxupquote{typeAlgorithm}} (\sphinxstyleliteralemphasis{\sphinxupquote{String}}) \textendash{} Method for calculate fractal dimensions: ‘SB’ for Sandbox, ‘BC’ for BoxCounting, ‘FSBC’ for fixed size box counting

\end{itemize}

\item[{Returns}] \leavevmode
\begin{description}
\item[{logR: Numpy array}] \leavevmode
logarithm of r/d

\item[{Indexzero: Integer}] \leavevmode
position of q=0 in Tq and Dq

\item[{Tq: Numpy array}] \leavevmode
mass exponents

\item[{Dq: Numpy array}] \leavevmode
fractal dimensions

\item[{lnMrq: Numpy 2D array}] \leavevmode
logarithm of number of nodes in boxes by radio

\item[{fitNessAverage: Numpy array}] \leavevmode
Average fitness across iterations

\item[{fitNessMax: Numpy array}] \leavevmode
Maximum fitness across iterations

\item[{fitNessMin: Numpy array:}] \leavevmode
Minimal fitness across iterations

\end{description}


\end{description}\end{quote}

\end{fulllineitems}

\index{calculateCenters() (in module Genetic.Genetic)@\spxentry{calculateCenters()}\spxextra{in module Genetic.Genetic}}

\begin{fulllineitems}
\phantomsection\label{\detokenize{Genetic:Genetic.Genetic.calculateCenters}}\pysiglinewithargsret{\sphinxcode{\sphinxupquote{Genetic.Genetic.}}\sphinxbfcode{\sphinxupquote{calculateCenters}}}{\emph{graph}, \emph{numNodes}, \emph{iterations}, \emph{sizePopulation}, \emph{radius}, \emph{distances}, \emph{percentCrossOver}, \emph{percentMutation}, \emph{listDegree}, \emph{maxDegree}, \emph{degreeOfBoring}}{}
Calculate centers with a random size between 20\% and 90\% of number of nodes

The genetic algorithm:
\begin{enumerate}
\def\theenumi{\arabic{enumi}}
\def\labelenumi{\theenumi .}
\makeatletter\def\p@enumii{\p@enumi \theenumi .}\makeatother
\item {} 
Generate a poblation of ramdom nodes as centers of the boxes

\item {} 
Evaluate each set of nodes with fitness funtion

\item {} 
Categorize poblation according fitness

\item {} 
Select two indivudues into population, then create a new individual

\item {} 
Remove worst individuals

\item {} 
Repeat 2 to 5 ultil a number of iterations or a boring degree

\end{enumerate}
\begin{quote}\begin{description}
\item[{Parameters}] \leavevmode\begin{itemize}
\item {} 
\sphinxstyleliteralstrong{\sphinxupquote{graph}} (\sphinxstyleliteralemphasis{\sphinxupquote{Snap PUN Graph.}}) \textendash{} Network.

\item {} 
\sphinxstyleliteralstrong{\sphinxupquote{numNodes}} (\sphinxstyleliteralemphasis{\sphinxupquote{Integer}}) \textendash{} Number of nodes in the network.

\item {} 
\sphinxstyleliteralstrong{\sphinxupquote{sizePopulation}} (\sphinxstyleliteralemphasis{\sphinxupquote{Integer}}) \textendash{} Size of population

\item {} 
\sphinxstyleliteralstrong{\sphinxupquote{radius}} (\sphinxstyleliteralemphasis{\sphinxupquote{Integer}}) \textendash{} Diameter of the network

\item {} 
\sphinxstyleliteralstrong{\sphinxupquote{distances}} (\sphinxstyleliteralemphasis{\sphinxupquote{Numpy 2D array of integers}}) \textendash{} Distance to other nodes

\item {} 
\sphinxstyleliteralstrong{\sphinxupquote{percentCrossOver}} (\sphinxstyleliteralemphasis{\sphinxupquote{Double}}) \textendash{} Percent (0 to 1) of individual select to cross

\item {} 
\sphinxstyleliteralstrong{\sphinxupquote{percentMutation}} (\sphinxstyleliteralemphasis{\sphinxupquote{Double}}) \textendash{} (0 to 1) of probability to apply mutation to an individual

\item {} 
\sphinxstyleliteralstrong{\sphinxupquote{listDegree}} (\sphinxstyleliteralemphasis{\sphinxupquote{Numpy 1D Array}}) \textendash{} List of degree all nodes

\item {} 
\sphinxstyleliteralstrong{\sphinxupquote{maxDegree}} (\sphinxstyleliteralemphasis{\sphinxupquote{Integer}}) \textendash{} Max degree in the network

\item {} 
\sphinxstyleliteralstrong{\sphinxupquote{degreeOfBoring}} (\sphinxstyleliteralemphasis{\sphinxupquote{Integer}}) \textendash{} Number of iterations of boring

\end{itemize}

\item[{Returns}] \leavevmode
\begin{description}
\item[{best: Numpy array}] \leavevmode
Best individual, select centers of boxes

\item[{Indexzero: Integer}] \leavevmode
position of q=0 in Tq and Dq

\item[{Tq: Numpy array}] \leavevmode
mass exponents

\item[{Dq: Numpy array}] \leavevmode
fractal dimensions

\item[{lnMrq: Numpy 2D array}] \leavevmode
logarithm of number of nodes in boxes by radio

\item[{fitNessAverage: Numpy array}] \leavevmode
Average fitness across iterations

\item[{fitNessMax: Numpy array}] \leavevmode
Maximum fitness across iterations

\item[{fitNessMin: Numpy array:}] \leavevmode
Minimal fitness across iterations

\end{description}


\end{description}\end{quote}

\end{fulllineitems}

\index{calculateCentersFixedSize() (in module Genetic.Genetic)@\spxentry{calculateCentersFixedSize()}\spxextra{in module Genetic.Genetic}}

\begin{fulllineitems}
\phantomsection\label{\detokenize{Genetic:Genetic.Genetic.calculateCentersFixedSize}}\pysiglinewithargsret{\sphinxcode{\sphinxupquote{Genetic.Genetic.}}\sphinxbfcode{\sphinxupquote{calculateCentersFixedSize}}}{\emph{graph}, \emph{numNodes}, \emph{iterations}, \emph{sizePopulation}, \emph{radius}, \emph{distances}, \emph{percentCrossOver}, \emph{percentMutation}, \emph{listDegree}, \emph{maxDegree}, \emph{sizeChromosome}, \emph{degreeOfBoring}}{}
Calculate centers with a specified size

The genetic algorithm:
\begin{enumerate}
\def\theenumi{\arabic{enumi}}
\def\labelenumi{\theenumi .}
\makeatletter\def\p@enumii{\p@enumi \theenumi .}\makeatother
\item {} 
Generate a poblation of ramdom nodes as centers of the boxes

\item {} 
Evaluate each set of nodes with fitness funtion

\item {} 
Categorize poblation according fitness

\item {} 
Select two indivudues into population, then create a new individual

\item {} 
Remove worst individuals

\item {} 
Repeat 2 to 5 ultil a number of iterations or a boring degree

\end{enumerate}
\begin{quote}\begin{description}
\item[{Parameters}] \leavevmode\begin{itemize}
\item {} 
\sphinxstyleliteralstrong{\sphinxupquote{graph}} (\sphinxstyleliteralemphasis{\sphinxupquote{Snap PUN Graph.}}) \textendash{} Network.

\item {} 
\sphinxstyleliteralstrong{\sphinxupquote{numNodes}} (\sphinxstyleliteralemphasis{\sphinxupquote{Integer}}) \textendash{} Number of nodes in the network.

\item {} 
\sphinxstyleliteralstrong{\sphinxupquote{sizePopulation}} (\sphinxstyleliteralemphasis{\sphinxupquote{Integer}}) \textendash{} Size of population

\item {} 
\sphinxstyleliteralstrong{\sphinxupquote{radius}} (\sphinxstyleliteralemphasis{\sphinxupquote{Integer}}) \textendash{} Diameter of the network

\item {} 
\sphinxstyleliteralstrong{\sphinxupquote{distances}} (\sphinxstyleliteralemphasis{\sphinxupquote{Numpy 2D array of integers}}) \textendash{} Distance to other nodes

\item {} 
\sphinxstyleliteralstrong{\sphinxupquote{percentCrossOver}} (\sphinxstyleliteralemphasis{\sphinxupquote{Double}}) \textendash{} Percent (0 to 1) of individual select to cross

\item {} 
\sphinxstyleliteralstrong{\sphinxupquote{Percent}} (\sphinxstyleliteralemphasis{\sphinxupquote{Double}}) \textendash{} (0 to 1) of probability to apply mutation to an individual

\item {} 
\sphinxstyleliteralstrong{\sphinxupquote{listDegree}} (\sphinxstyleliteralemphasis{\sphinxupquote{Numpy 1D Array}}) \textendash{} List of degree all nodes

\item {} 
\sphinxstyleliteralstrong{\sphinxupquote{maxDegree}} (\sphinxstyleliteralemphasis{\sphinxupquote{Integer}}) \textendash{} Max degree in the network

\item {} 
\sphinxstyleliteralstrong{\sphinxupquote{sizeChromosome}} (\sphinxstyleliteralemphasis{\sphinxupquote{Integer}}) \textendash{} Number of centers of boxes

\item {} 
\sphinxstyleliteralstrong{\sphinxupquote{degreeOfBoring}} (\sphinxstyleliteralemphasis{\sphinxupquote{Integer}}) \textendash{} Number of iterations of boring

\end{itemize}

\item[{Returns}] \leavevmode
\begin{description}
\item[{best: Numpy array}] \leavevmode
Best individual, select centers of boxes

\end{description}


\end{description}\end{quote}

\end{fulllineitems}

\index{calculateFitness() (in module Genetic.Genetic)@\spxentry{calculateFitness()}\spxextra{in module Genetic.Genetic}}

\begin{fulllineitems}
\phantomsection\label{\detokenize{Genetic:Genetic.Genetic.calculateFitness}}\pysiglinewithargsret{\sphinxcode{\sphinxupquote{Genetic.Genetic.}}\sphinxbfcode{\sphinxupquote{calculateFitness}}}{\emph{graph}, \emph{chromosome}, \emph{radius}, \emph{distances}, \emph{listDegree}, \emph{maxDegree}}{}
Calculate fitness from a select node centers in a network

Fitness is the average between distances of the centers and the average the degrees , the centers can be of different size
\begin{quote}\begin{description}
\item[{Parameters}] \leavevmode\begin{itemize}
\item {} 
\sphinxstyleliteralstrong{\sphinxupquote{graph}} (\sphinxstyleliteralemphasis{\sphinxupquote{Snap PUN Graph.}}) \textendash{} Network.

\item {} 
\sphinxstyleliteralstrong{\sphinxupquote{chromosome}} (\sphinxstyleliteralemphasis{\sphinxupquote{Numpy array of integers}}) \textendash{} Centers

\item {} 
\sphinxstyleliteralstrong{\sphinxupquote{radius}} (\sphinxstyleliteralemphasis{\sphinxupquote{Integer}}) \textendash{} Diameter of the network

\item {} 
\sphinxstyleliteralstrong{\sphinxupquote{distances}} (\sphinxstyleliteralemphasis{\sphinxupquote{Numpy 2D array of integers}}) \textendash{} Distance between all nodes

\item {} 
\sphinxstyleliteralstrong{\sphinxupquote{listDegree}} (\sphinxstyleliteralemphasis{\sphinxupquote{Numpy 1D Array}}) \textendash{} List of degree all nodes

\item {} 
\sphinxstyleliteralstrong{\sphinxupquote{maxDegree}} (\sphinxstyleliteralemphasis{\sphinxupquote{Integer}}) \textendash{} Max degree in the network

\end{itemize}

\item[{Returns}] \leavevmode
Fitness of the centers

\item[{Type}] \leavevmode
Double

\end{description}\end{quote}

\end{fulllineitems}



\subsection{Example}
\label{\detokenize{Genetic:example}}
\fvset{hllines={, 2, 7,}}%
\begin{sphinxVerbatim}[commandchars=\\\{\}]
\PYG{k+kn}{import} \PYG{n+nn}{sys}
\PYG{k+kn}{import} \PYG{n+nn}{lib.snap} \PYG{k+kn}{as} \PYG{n+nn}{snap}
\PYG{k+kn}{import} \PYG{n+nn}{Genetic.Genetic} \PYG{k+kn}{as} \PYG{n+nn}{Genetic}
\PYG{k+kn}{import} \PYG{n+nn}{numpy}

\PYG{n}{minq} \PYG{o}{=} \PYG{o}{\PYGZhy{}}\PYG{l+m+mi}{10}
\PYG{n}{maxq} \PYG{o}{=} \PYG{l+m+mi}{10}
\PYG{n}{sizePopulation} \PYG{o}{=} \PYG{l+m+mi}{200}
\PYG{n}{percentCrossOver} \PYG{o}{=} \PYG{l+m+mf}{0.4}
\PYG{n}{percentMutation} \PYG{o}{=} \PYG{l+m+mf}{0.05}
\PYG{n}{degreeOfBoring} \PYG{o}{=} \PYG{l+m+mi}{20}
\PYG{n}{Rnd} \PYG{o}{=} \PYG{n}{snap}\PYG{o}{.}\PYG{n}{TRnd}\PYG{p}{(}\PYG{l+m+mi}{1}\PYG{p}{,}\PYG{l+m+mi}{0}\PYG{p}{)}
\PYG{n}{graph} \PYG{o}{=} \PYG{n}{snap}\PYG{o}{.}\PYG{n}{GenPrefAttach}\PYG{p}{(}\PYG{l+m+mi}{10000}\PYG{p}{,} \PYG{l+m+mi}{10}\PYG{p}{,}\PYG{n}{Rnd}\PYG{p}{)}  \PYG{c+c1}{\PYGZsh{}ScaleFree with 10 edges per node}

\PYG{n}{logR}\PYG{p}{,} \PYG{n}{Indexzero}\PYG{p}{,}\PYG{n}{Tq}\PYG{p}{,} \PYG{n}{Dq}\PYG{p}{,} \PYG{n}{lnMrq}\PYG{p}{,}\PYG{n}{fitNessAverage}\PYG{p}{,}\PYG{n}{fitNessMax}\PYG{p}{,}\PYG{n}{fitNessMin} \PYG{o}{=} \PYG{n}{Genetic}\PYG{o}{.}\PYG{n}{Genetic}\PYG{p}{(}\PYG{n}{graph}\PYG{p}{,}\PYG{n}{minq}\PYG{p}{,}\PYG{n}{maxq}\PYG{p}{,}\PYG{n}{sizePopulation}\PYG{p}{,}\PYG{n}{iterations}\PYG{p}{,} \PYG{n}{percentCrossOver}\PYG{p}{,} \PYG{n}{percentMutation}\PYG{p}{,}\PYG{n}{degreeOfBoring}\PYG{p}{,} \PYG{l+s+s1}{\PYGZsq{}}\PYG{l+s+s1}{SB}\PYG{l+s+s1}{\PYGZsq{}}\PYG{p}{)}
\end{sphinxVerbatim}
\sphinxresetverbatimhllines


\section{SandBox Algorithm}
\label{\detokenize{SBAlgorithm:sandbox-algorithm}}\label{\detokenize{SBAlgorithm::doc}}
This package provides multifractal analysis with SandBox Algorithm proposed in journal article: Determination of multifractal dimensions of complex networks by means of the sandbox algorithm DOI: 10.1063/1.4907557


\subsection{SBAlgorithm}
\label{\detokenize{SBAlgorithm:module-SBAlgorithm.SBAlgorithm}}\label{\detokenize{SBAlgorithm:sbalgorithm}}\index{SBAlgorithm.SBAlgorithm (module)@\spxentry{SBAlgorithm.SBAlgorithm}\spxextra{module}}\index{SBAlgorithm() (in module SBAlgorithm.SBAlgorithm)@\spxentry{SBAlgorithm()}\spxextra{in module SBAlgorithm.SBAlgorithm}}

\begin{fulllineitems}
\phantomsection\label{\detokenize{SBAlgorithm:SBAlgorithm.SBAlgorithm.SBAlgorithm}}\pysiglinewithargsret{\sphinxcode{\sphinxupquote{SBAlgorithm.SBAlgorithm.}}\sphinxbfcode{\sphinxupquote{SBAlgorithm}}}{\emph{g}, \emph{minq}, \emph{maxq}, \emph{percentSandBox}, \emph{repetitions}, \emph{centerNodes=array({[}{]}}, \emph{dtype=float64)}}{}
Calculate fractal dimension with SandBox method

Inputs are parameters to configure algorithm behaviour.
\begin{quote}\begin{description}
\item[{Parameters}] \leavevmode\begin{itemize}
\item {} 
\sphinxstyleliteralstrong{\sphinxupquote{g}} (\sphinxstyleliteralemphasis{\sphinxupquote{Snap PUN Graph.}}) \textendash{} Network.

\item {} 
\sphinxstyleliteralstrong{\sphinxupquote{minq}} \textendash{} Minimum value of q

\item {} 
\sphinxstyleliteralstrong{\sphinxupquote{minq}} \textendash{} Maximum value of q

\item {} 
\sphinxstyleliteralstrong{\sphinxupquote{percentSandBox}} (\sphinxstyleliteralemphasis{\sphinxupquote{Double}}) \textendash{} Number of combinations of center nodes. This value is a percent of the total nodes

\item {} 
\sphinxstyleliteralstrong{\sphinxupquote{repetitions}} (\sphinxstyleliteralemphasis{\sphinxupquote{Integer}}) \textendash{} Number of repetitions of algorithm

\item {} 
\sphinxstyleliteralstrong{\sphinxupquote{CenterNodes}} (\sphinxstyleliteralemphasis{\sphinxupquote{Numpy 1D Array}}) \textendash{} Calculated center. If this is null, then the centers are calculated

\end{itemize}

\item[{Returns}] \leavevmode
\begin{description}
\item[{logR: Numpy array}] \leavevmode
logarithm of r/d

\item[{Indexzero: Integer}] \leavevmode
position of q=0 in Tq and Dq

\item[{Tq: Numpy array}] \leavevmode
mass exponents

\item[{Dq: Numpy array}] \leavevmode
fractal dimensions

\item[{lnMrq: Numpy 2D array}] \leavevmode
logarithm of number of nodes in boxes by radio

\end{description}


\end{description}\end{quote}

\end{fulllineitems}



\subsection{Example}
\label{\detokenize{SBAlgorithm:example}}
\fvset{hllines={, 2, 7,}}%
\begin{sphinxVerbatim}[commandchars=\\\{\}]
\PYG{k+kn}{import} \PYG{n+nn}{sys}
\PYG{k+kn}{import} \PYG{n+nn}{lib.snap} \PYG{k+kn}{as} \PYG{n+nn}{snap}
\PYG{k+kn}{import} \PYG{n+nn}{SBAlgorithm.SBAlgorithm} \PYG{k+kn}{as} \PYG{n+nn}{SBAlgorithm}
\PYG{k+kn}{import} \PYG{n+nn}{numpy}

\PYG{n}{minq} \PYG{o}{=} \PYG{o}{\PYGZhy{}}\PYG{l+m+mi}{10}
\PYG{n}{maxq} \PYG{o}{=} \PYG{l+m+mi}{10}
\PYG{n}{percentOfSandBoxes} \PYG{o}{=} \PYG{l+m+mf}{0.6}
\PYG{n}{repetitionsSB} \PYG{o}{=} \PYG{l+m+mi}{50}
\PYG{n}{Rnd} \PYG{o}{=} \PYG{n}{snap}\PYG{o}{.}\PYG{n}{TRnd}\PYG{p}{(}\PYG{l+m+mi}{1}\PYG{p}{,}\PYG{l+m+mi}{0}\PYG{p}{)}
\PYG{n}{graph} \PYG{o}{=} \PYG{n}{snap}\PYG{o}{.}\PYG{n}{GenPrefAttach}\PYG{p}{(}\PYG{l+m+mi}{10000}\PYG{p}{,} \PYG{l+m+mi}{10}\PYG{p}{,}\PYG{n}{Rnd}\PYG{p}{)}  \PYG{c+c1}{\PYGZsh{}ScaleFree with 10 edges per node}

\PYG{n}{logRB}\PYG{p}{,} \PYG{n}{IndexzeroB}\PYG{p}{,}\PYG{n}{TqB}\PYG{p}{,} \PYG{n}{DqB}\PYG{p}{,} \PYG{n}{lnMrqB} \PYG{o}{=} \PYG{n}{SBAlgorithm}\PYG{o}{.}\PYG{n}{SBAlgorithm}\PYG{p}{(}\PYG{n}{graph}\PYG{p}{,}\PYG{n}{minq}\PYG{p}{,}\PYG{n}{maxq}\PYG{p}{,}\PYG{n}{percentOfSandBoxes}\PYG{p}{,}\PYG{n}{repetitionsSB}\PYG{p}{)}
\end{sphinxVerbatim}
\sphinxresetverbatimhllines


\section{Simulated Annealing strategy}
\label{\detokenize{SimulatedAnnealing:simulated-annealing-strategy}}\label{\detokenize{SimulatedAnnealing::doc}}
Simulated annealing algorithm for multifractal analysis


\subsection{SimulatedAnnealing.SimulatedAnnealing module}
\label{\detokenize{SimulatedAnnealing:module-SimulatedAnnealing.SimulatedAnnealing}}\label{\detokenize{SimulatedAnnealing:simulatedannealing-simulatedannealing-module}}\index{SimulatedAnnealing.SimulatedAnnealing (module)@\spxentry{SimulatedAnnealing.SimulatedAnnealing}\spxextra{module}}\index{SA() (in module SimulatedAnnealing.SimulatedAnnealing)@\spxentry{SA()}\spxextra{in module SimulatedAnnealing.SimulatedAnnealing}}

\begin{fulllineitems}
\phantomsection\label{\detokenize{SimulatedAnnealing:SimulatedAnnealing.SimulatedAnnealing.SA}}\pysiglinewithargsret{\sphinxcode{\sphinxupquote{SimulatedAnnealing.SimulatedAnnealing.}}\sphinxbfcode{\sphinxupquote{SA}}}{\emph{g}, \emph{minq}, \emph{maxq}, \emph{percentNodes}, \emph{sizePopulation}, \emph{Kmax}, \emph{typeAlgorithm}}{}
The simulated annealing algorithm
\begin{quote}\begin{description}
\item[{Parameters}] \leavevmode\begin{itemize}
\item {} 
\sphinxstyleliteralstrong{\sphinxupquote{graph}} (\sphinxstyleliteralemphasis{\sphinxupquote{Snap PUN Graph.}}) \textendash{} Network.

\item {} 
\sphinxstyleliteralstrong{\sphinxupquote{sizePopulation}} (\sphinxstyleliteralemphasis{\sphinxupquote{Integer}}) \textendash{} Size of population

\item {} 
\sphinxstyleliteralstrong{\sphinxupquote{Kmax}} (\sphinxstyleliteralemphasis{\sphinxupquote{Integer}}) \textendash{} Initial temperature

\item {} 
\sphinxstyleliteralstrong{\sphinxupquote{typeAlgorithm}} (\sphinxstyleliteralemphasis{\sphinxupquote{String}}) \textendash{} Method for calculate fractal dimensions: ‘SB’ for Sandbox, ‘BC’ for BoxCounting, ‘FSBC’ for fixed size box counting

\end{itemize}

\item[{Returns}] \leavevmode
\begin{description}
\item[{logR: Numpy array}] \leavevmode
logarithm of r/d

\item[{Indexzero: Integer}] \leavevmode
position of q=0 in Tq and Dq

\item[{Tq: Numpy array}] \leavevmode
mass exponents

\item[{Dq: Numpy array}] \leavevmode
fractal dimensions

\item[{lnMrq: Numpy 2D array}] \leavevmode
logarithm of number of nodes in boxes by radio

\end{description}


\end{description}\end{quote}

\end{fulllineitems}

\index{calculateCenters() (in module SimulatedAnnealing.SimulatedAnnealing)@\spxentry{calculateCenters()}\spxextra{in module SimulatedAnnealing.SimulatedAnnealing}}

\begin{fulllineitems}
\phantomsection\label{\detokenize{SimulatedAnnealing:SimulatedAnnealing.SimulatedAnnealing.calculateCenters}}\pysiglinewithargsret{\sphinxcode{\sphinxupquote{SimulatedAnnealing.SimulatedAnnealing.}}\sphinxbfcode{\sphinxupquote{calculateCenters}}}{\emph{graph}, \emph{numNodes}, \emph{percentNodes}, \emph{Kmax}, \emph{d}, \emph{distances}, \emph{listID}, \emph{listDegree}, \emph{totalRemoved=0}}{}
Calculate centers with a specified size

The genetic algorithm:
\begin{enumerate}
\def\theenumi{\arabic{enumi}}
\def\labelenumi{\theenumi .}
\makeatletter\def\p@enumii{\p@enumi \theenumi .}\makeatother
\item {} 
Generate a poblation of ramdom nodes as centers of the boxes

\item {} 
Select a neighbor state

\item {} 
Select this state according its fitness and global temperature

\end{enumerate}
\begin{quote}\begin{description}
\item[{Parameters}] \leavevmode\begin{itemize}
\item {} 
\sphinxstyleliteralstrong{\sphinxupquote{graph}} (\sphinxstyleliteralemphasis{\sphinxupquote{Snap PUN Graph.}}) \textendash{} Network.

\item {} 
\sphinxstyleliteralstrong{\sphinxupquote{numNodes}} (\sphinxstyleliteralemphasis{\sphinxupquote{Integer}}) \textendash{} Number of nodes in the network.

\item {} 
\sphinxstyleliteralstrong{\sphinxupquote{percentNodes}} (\sphinxstyleliteralemphasis{\sphinxupquote{Integer}}) \textendash{} Size of population

\item {} 
\sphinxstyleliteralstrong{\sphinxupquote{Kmax}} (\sphinxstyleliteralemphasis{\sphinxupquote{Integer}}) \textendash{} System temperature

\item {} 
\sphinxstyleliteralstrong{\sphinxupquote{lisID}} (\sphinxstyleliteralemphasis{\sphinxupquote{Numpy 2D array of integers}}) \textendash{} ID of nodes

\item {} 
\sphinxstyleliteralstrong{\sphinxupquote{listDegree}} (\sphinxstyleliteralemphasis{\sphinxupquote{Numpy 2D array of integers}}) \textendash{} List of degree all nodes

\item {} 
\sphinxstyleliteralstrong{\sphinxupquote{totalRemoved}} (\sphinxstyleliteralemphasis{\sphinxupquote{Integer}}) \textendash{} Number of nodes in solution, only if you want apply robusness analysis

\end{itemize}

\item[{Returns}] \leavevmode
\begin{description}
\item[{currentState: Numpy array}] \leavevmode
current individual, select centers of boxes

\end{description}


\end{description}\end{quote}

\end{fulllineitems}

\index{calculateFitness() (in module SimulatedAnnealing.SimulatedAnnealing)@\spxentry{calculateFitness()}\spxextra{in module SimulatedAnnealing.SimulatedAnnealing}}

\begin{fulllineitems}
\phantomsection\label{\detokenize{SimulatedAnnealing:SimulatedAnnealing.SimulatedAnnealing.calculateFitness}}\pysiglinewithargsret{\sphinxcode{\sphinxupquote{SimulatedAnnealing.SimulatedAnnealing.}}\sphinxbfcode{\sphinxupquote{calculateFitness}}}{\emph{g}, \emph{element}, \emph{radius}, \emph{distances}, \emph{listID}, \emph{listDegree}}{}
Calculate fitness from a select node centers in a network

Fitness is the average between distances of the centers and the average the degrees , the centers can be of different size
\begin{quote}\begin{description}
\item[{Parameters}] \leavevmode\begin{itemize}
\item {} 
\sphinxstyleliteralstrong{\sphinxupquote{graph}} (\sphinxstyleliteralemphasis{\sphinxupquote{Snap PUN Graph.}}) \textendash{} Network.

\item {} 
\sphinxstyleliteralstrong{\sphinxupquote{chromosome}} (\sphinxstyleliteralemphasis{\sphinxupquote{Numpy array of integers}}) \textendash{} Centers

\item {} 
\sphinxstyleliteralstrong{\sphinxupquote{radius}} (\sphinxstyleliteralemphasis{\sphinxupquote{Integer}}) \textendash{} Diameter of the network

\item {} 
\sphinxstyleliteralstrong{\sphinxupquote{distances}} (\sphinxstyleliteralemphasis{\sphinxupquote{Numpy 2D array of integers}}) \textendash{} Distance between all nodes

\item {} 
\sphinxstyleliteralstrong{\sphinxupquote{listDegree}} (\sphinxstyleliteralemphasis{\sphinxupquote{Numpy 1D Array}}) \textendash{} List of degree all nodes

\item {} 
\sphinxstyleliteralstrong{\sphinxupquote{maxDegree}} (\sphinxstyleliteralemphasis{\sphinxupquote{Integer}}) \textendash{} Max degree in the network

\end{itemize}

\item[{Returns}] \leavevmode
Fitness of the centers

\item[{Type}] \leavevmode
Double

\end{description}\end{quote}

\end{fulllineitems}

\index{createNeighbors() (in module SimulatedAnnealing.SimulatedAnnealing)@\spxentry{createNeighbors()}\spxextra{in module SimulatedAnnealing.SimulatedAnnealing}}

\begin{fulllineitems}
\phantomsection\label{\detokenize{SimulatedAnnealing:SimulatedAnnealing.SimulatedAnnealing.createNeighbors}}\pysiglinewithargsret{\sphinxcode{\sphinxupquote{SimulatedAnnealing.SimulatedAnnealing.}}\sphinxbfcode{\sphinxupquote{createNeighbors}}}{\emph{node}, \emph{numNodes}, \emph{distances}}{}
Calculate fitness from a select node centers in a network

Fitness is the average between distances of the centers and the average the degrees , the centers can be of different size
\begin{quote}\begin{description}
\item[{Parameters}] \leavevmode\begin{itemize}
\item {} 
\sphinxstyleliteralstrong{\sphinxupquote{node}} (\sphinxstyleliteralemphasis{\sphinxupquote{Integer.}}) \textendash{} ID of node in the network.

\item {} 
\sphinxstyleliteralstrong{\sphinxupquote{numNodes}} (\sphinxstyleliteralemphasis{\sphinxupquote{Integer}}) \textendash{} Number of nodes

\item {} 
\sphinxstyleliteralstrong{\sphinxupquote{distances}} (\sphinxstyleliteralemphasis{\sphinxupquote{Numpy 2D array of integers}}) \textendash{} Distance between all nodes

\end{itemize}

\item[{Returns}] \leavevmode
neighbors. Numpy ID array

\item[{Type}] \leavevmode
Double

\end{description}\end{quote}

\end{fulllineitems}



\subsection{Example}
\label{\detokenize{SimulatedAnnealing:example}}
\fvset{hllines={, 2, 7,}}%
\begin{sphinxVerbatim}[commandchars=\\\{\}]
\PYG{k+kn}{import} \PYG{n+nn}{sys}
\PYG{k+kn}{import} \PYG{n+nn}{lib.snap} \PYG{k+kn}{as} \PYG{n+nn}{snap}
\PYG{k+kn}{import} \PYG{n+nn}{SimulatedAnnealing.SimulatedAnnealing} \PYG{k+kn}{as} \PYG{n+nn}{SimulatedAnnealing}
\PYG{k+kn}{import} \PYG{n+nn}{numpy}

\PYG{n}{minq} \PYG{o}{=} \PYG{o}{\PYGZhy{}}\PYG{l+m+mi}{10}
\PYG{n}{maxq} \PYG{o}{=} \PYG{l+m+mi}{10}
\PYG{n}{sizePopulation} \PYG{o}{=} \PYG{l+m+mi}{200}
\PYG{n}{Kmax} \PYG{o}{=} \PYG{l+m+mi}{1500}
\PYG{n}{Rnd} \PYG{o}{=} \PYG{n}{snap}\PYG{o}{.}\PYG{n}{TRnd}\PYG{p}{(}\PYG{l+m+mi}{1}\PYG{p}{,}\PYG{l+m+mi}{0}\PYG{p}{)}
\PYG{n}{graph} \PYG{o}{=} \PYG{n}{snap}\PYG{o}{.}\PYG{n}{GenPrefAttach}\PYG{p}{(}\PYG{l+m+mi}{10000}\PYG{p}{,} \PYG{l+m+mi}{10}\PYG{p}{,}\PYG{n}{Rnd}\PYG{p}{)}  \PYG{c+c1}{\PYGZsh{}ScaleFree with 10 edges per node}

\PYG{n}{logRD}\PYG{p}{,} \PYG{n}{IndexzeroD}\PYG{p}{,}\PYG{n}{TqD}\PYG{p}{,} \PYG{n}{DqD}\PYG{p}{,} \PYG{n}{lnMrqD} \PYG{o}{=} \PYG{n}{SimulatedAnnealing}\PYG{o}{.}\PYG{n}{SA}\PYG{p}{(}\PYG{n}{graph}\PYG{p}{,}\PYG{n}{minq}\PYG{p}{,}\PYG{n}{maxq}\PYG{p}{,}\PYG{n}{percentOfSandBoxes}\PYG{p}{,}\PYG{n}{sizePopulation}\PYG{p}{,} \PYG{n}{Kmax}\PYG{p}{,} \PYG{l+s+s1}{\PYGZsq{}}\PYG{l+s+s1}{BC}\PYG{l+s+s1}{\PYGZsq{}}\PYG{p}{)}
\end{sphinxVerbatim}
\sphinxresetverbatimhllines


\section{robustness package}
\label{\detokenize{robustness:robustness-package}}\label{\detokenize{robustness::doc}}

\subsection{Submodules}
\label{\detokenize{robustness:submodules}}

\subsection{robustness.robustness module}
\label{\detokenize{robustness:module-robustness.robustness}}\label{\detokenize{robustness:robustness-robustness-module}}\index{robustness.robustness (module)@\spxentry{robustness.robustness}\spxextra{module}}\index{robustness\_analysis() (in module robustness.robustness)@\spxentry{robustness\_analysis()}\spxextra{in module robustness.robustness}}

\begin{fulllineitems}
\phantomsection\label{\detokenize{robustness:robustness.robustness.robustness_analysis}}\pysiglinewithargsret{\sphinxcode{\sphinxupquote{robustness.robustness.}}\sphinxbfcode{\sphinxupquote{robustness\_analysis}}}{\emph{graph}, \emph{typeRemoval}, \emph{minq}, \emph{maxq}, \emph{percentSandBox}, \emph{repetitions}, \emph{temperature=0}, \emph{sizePopulation=0}, \emph{iterationsGenetic=0}, \emph{percentCrossOver=0}, \emph{percentMutation=0}, \emph{degreeOfBoring=0}, \emph{percentOfNodes=0.1}, \emph{initialPercent=0.1}, \emph{finalPercent=1.0}, \emph{iteracionPercent=0.1}, \emph{nameFile='none'}}{}
\end{fulllineitems}



\subsection{Module contents}
\label{\detokenize{robustness:module-robustness}}\label{\detokenize{robustness:module-contents}}\index{robustness (module)@\spxentry{robustness}\spxextra{module}}

\section{utils package}
\label{\detokenize{utils:utils-package}}\label{\detokenize{utils::doc}}

\subsection{Submodules}
\label{\detokenize{utils:submodules}}

\subsection{utils.utils module}
\label{\detokenize{utils:module-utils.utils}}\label{\detokenize{utils:utils-utils-module}}\index{utils.utils (module)@\spxentry{utils.utils}\spxextra{module}}\index{copyGraph() (in module utils.utils)@\spxentry{copyGraph()}\spxextra{in module utils.utils}}

\begin{fulllineitems}
\phantomsection\label{\detokenize{utils:utils.utils.copyGraph}}\pysiglinewithargsret{\sphinxcode{\sphinxupquote{utils.utils.}}\sphinxbfcode{\sphinxupquote{copyGraph}}}{\emph{graph}}{}
\end{fulllineitems}

\index{getAdjacenceMatriz() (in module utils.utils)@\spxentry{getAdjacenceMatriz()}\spxextra{in module utils.utils}}

\begin{fulllineitems}
\phantomsection\label{\detokenize{utils:utils.utils.getAdjacenceMatriz}}\pysiglinewithargsret{\sphinxcode{\sphinxupquote{utils.utils.}}\sphinxbfcode{\sphinxupquote{getAdjacenceMatriz}}}{\emph{distances}, \emph{numNodes}}{}
\end{fulllineitems}

\index{getAveragePathLength() (in module utils.utils)@\spxentry{getAveragePathLength()}\spxextra{in module utils.utils}}

\begin{fulllineitems}
\phantomsection\label{\detokenize{utils:utils.utils.getAveragePathLength}}\pysiglinewithargsret{\sphinxcode{\sphinxupquote{utils.utils.}}\sphinxbfcode{\sphinxupquote{getAveragePathLength}}}{\emph{graph}}{}
\end{fulllineitems}

\index{getDistancesMatrix() (in module utils.utils)@\spxentry{getDistancesMatrix()}\spxextra{in module utils.utils}}

\begin{fulllineitems}
\phantomsection\label{\detokenize{utils:utils.utils.getDistancesMatrix}}\pysiglinewithargsret{\sphinxcode{\sphinxupquote{utils.utils.}}\sphinxbfcode{\sphinxupquote{getDistancesMatrix}}}{\emph{graph}, \emph{numNodes}, \emph{listID}}{}
\end{fulllineitems}

\index{getOrderedClosenessCentrality() (in module utils.utils)@\spxentry{getOrderedClosenessCentrality()}\spxextra{in module utils.utils}}

\begin{fulllineitems}
\phantomsection\label{\detokenize{utils:utils.utils.getOrderedClosenessCentrality}}\pysiglinewithargsret{\sphinxcode{\sphinxupquote{utils.utils.}}\sphinxbfcode{\sphinxupquote{getOrderedClosenessCentrality}}}{\emph{graph}, \emph{N}}{}
\end{fulllineitems}

\index{getSizeOfGiantComponent() (in module utils.utils)@\spxentry{getSizeOfGiantComponent()}\spxextra{in module utils.utils}}

\begin{fulllineitems}
\phantomsection\label{\detokenize{utils:utils.utils.getSizeOfGiantComponent}}\pysiglinewithargsret{\sphinxcode{\sphinxupquote{utils.utils.}}\sphinxbfcode{\sphinxupquote{getSizeOfGiantComponent}}}{\emph{graph}}{}
\end{fulllineitems}

\index{linealRegresssion() (in module utils.utils)@\spxentry{linealRegresssion()}\spxextra{in module utils.utils}}

\begin{fulllineitems}
\phantomsection\label{\detokenize{utils:utils.utils.linealRegresssion}}\pysiglinewithargsret{\sphinxcode{\sphinxupquote{utils.utils.}}\sphinxbfcode{\sphinxupquote{linealRegresssion}}}{\emph{x}, \emph{y}}{}
\end{fulllineitems}

\index{removeNodes() (in module utils.utils)@\spxentry{removeNodes()}\spxextra{in module utils.utils}}

\begin{fulllineitems}
\phantomsection\label{\detokenize{utils:utils.utils.removeNodes}}\pysiglinewithargsret{\sphinxcode{\sphinxupquote{utils.utils.}}\sphinxbfcode{\sphinxupquote{removeNodes}}}{\emph{graph}, \emph{typeRemoval}, \emph{p}, \emph{numberNodesToRemove}, \emph{ClosenessCentrality}, \emph{listID}, \emph{nodesToRemove=array({[}{]}}, \emph{dtype=float64)}}{}
\end{fulllineitems}



\subsection{Module contents}
\label{\detokenize{utils:module-utils}}\label{\detokenize{utils:module-contents}}\index{utils (module)@\spxentry{utils}\spxextra{module}}

\chapter{Requeriments}
\label{\detokenize{index:requeriments}}\begin{itemize}
\item {} 
numpy \textgreater{}= 1.12.1

\item {} 
snap \textgreater{}= 0.5

\item {} 
matplotlib \textgreater{}= 2.0

\item {} 
python = 2.7

\end{itemize}


\chapter{Indices and tables}
\label{\detokenize{index:indices-and-tables}}\begin{itemize}
\item {} 
\DUrole{xref,std,std-ref}{genindex}

\item {} 
\DUrole{xref,std,std-ref}{modindex}

\item {} 
\DUrole{xref,std,std-ref}{search}

\end{itemize}


\renewcommand{\indexname}{Python Module Index}
\begin{sphinxtheindex}
\let\bigletter\sphinxstyleindexlettergroup
\bigletter{b}
\item\relax\sphinxstyleindexentry{BCAlgorithm.BCAlgorithm}\sphinxstyleindexpageref{BCAlgorithm:\detokenize{module-BCAlgorithm.BCAlgorithm}}
\indexspace
\bigletter{c}
\item\relax\sphinxstyleindexentry{CBBAlgorithm}\sphinxstyleindexpageref{CBBAlgorithm:\detokenize{module-CBBAlgorithm}}
\item\relax\sphinxstyleindexentry{CBBAlgorithm.CBBAlgorithm}\sphinxstyleindexpageref{CBBAlgorithm:\detokenize{module-CBBAlgorithm.CBBAlgorithm}}
\indexspace
\bigletter{f}
\item\relax\sphinxstyleindexentry{FSBCAlgorithm.FSBCAlgorithm}\sphinxstyleindexpageref{FSBCAlgorithm:\detokenize{module-FSBCAlgorithm.FSBCAlgorithm}}
\indexspace
\bigletter{g}
\item\relax\sphinxstyleindexentry{Genetic.Genetic}\sphinxstyleindexpageref{Genetic:\detokenize{module-Genetic.Genetic}}
\indexspace
\bigletter{r}
\item\relax\sphinxstyleindexentry{robustness}\sphinxstyleindexpageref{robustness:\detokenize{module-robustness}}
\item\relax\sphinxstyleindexentry{robustness.robustness}\sphinxstyleindexpageref{robustness:\detokenize{module-robustness.robustness}}
\indexspace
\bigletter{s}
\item\relax\sphinxstyleindexentry{SBAlgorithm.SBAlgorithm}\sphinxstyleindexpageref{SBAlgorithm:\detokenize{module-SBAlgorithm.SBAlgorithm}}
\item\relax\sphinxstyleindexentry{SimulatedAnnealing.SimulatedAnnealing}\sphinxstyleindexpageref{SimulatedAnnealing:\detokenize{module-SimulatedAnnealing.SimulatedAnnealing}}
\indexspace
\bigletter{u}
\item\relax\sphinxstyleindexentry{utils}\sphinxstyleindexpageref{utils:\detokenize{module-utils}}
\item\relax\sphinxstyleindexentry{utils.utils}\sphinxstyleindexpageref{utils:\detokenize{module-utils.utils}}
\end{sphinxtheindex}

\renewcommand{\indexname}{Index}
\printindex
\end{document}